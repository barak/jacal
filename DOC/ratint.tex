\documentstyle [eqalign] {article}
\def\({\left(}
\def\){\right)}
\title{Solving Algebraic First Order Differential Equations}
\author{Aubrey Jaffer}
\date{December 6, 1990}    % Deleting this command produces today's date.

\begin{document}           % End of preamble and beginning of text.

\maketitle                 % Produces the title.
\newtheorem{lemma}{Lemma}

The derivative of any algebraic expression is algebraic.  First solve
the problem of finding antiderivatives where the solution is a
rational expression.  Work backwards from the form of the solution to
completely characterize those derivatives which can lead to the
algebraic solution.

\section{Rational Function Differentiation}

Let $$y=\prod_{i=1}^k p_i(x)^{n_i} \label{product}$$ be a rational
function of x where the polynomials $p_i(x)$ are squarefree and
mutually relatively prime.

The derivative of $y$ is
\begin{equation}
{y'} = {x'} \sum_{i=1}^k n_i p_i(x)^{n_i-1} {p_i'}(x)
\prod_{j \neq i} p_j(x)^{n_j}\label{product'}
\end{equation}

\begin{lemma}
\label{comfact}
The expression $\sum_{i=1}^k n_i {p_i'}(x) \prod_{j \neq i} p_j(x)$
has no factors in common with $p_i(x)$.
\end{lemma}
Assume that the expression has a common factor $p_h(x)$. Then
$$p_h(x) \quad\mbox{divides}\quad
\sum_{i=1}^k n_i {p_i'}(x) \prod_{j \neq i} p_j(x)$$.

Now, $p_h(x)$ divides all terms for $i \neq h$ and since it divides
the whole sum, $p_h(x)$ must divide the remaining term $n_h p_h'(x)
\prod_{j \neq h} p_j(x)$.  But, from the above conditions, $p_h(x)$
does not divide $p_h'(x)$ [$p_h(x)$ is squarefree] and $p_h(x)$ does
not divide $p_h(x)$ for $j \neq h$ [relatively prime condition].

\section{Rational Function Integration}

Now
\begin{equation}
{y'} = {x'} \( \prod_{i=1}^k p_i(x)^{n_i-1} \)
\sum_{i=1}^k n_i {p_i'}(x) \prod_{j \neq i} p_j(x)\label{'fact}
\end{equation}
There are no common factors between the sum and product terms of
equation~\ref{'fact} because of the relatively prime condition of
equation~\ref{product} and because of Lemma~\ref{comfact}.  Hence,
this equation cannot be reduced and is canonical.

Split equation \ref{'fact} into factors with positive and negative
exponents and renumber $i$ to be negative when $n_i$ is negative, giving
\begin{equation}
{y'} \prod_{-i} p_i(x)^{-n_i+1} =
{x'} \( \prod_{+i} p_i(x)^{n_i-1} \)
\sum_{i} n_i {p_i'}(x) \prod_{j \neq i} p_j(x)\label{renum}
\end{equation}

Now to integrate equation~\ref{renum} note that exponents $-n_i+1 >
1$ because $n_i~<~0$.  Hence $\prod_{-i} p_i(x)^{-n_i+1}$ can factored
(easily in fact by squarefree factorization).  Now segregate the terms
in the sum of equation~\ref{'fact} as well.
$$\eqalign{&\sum_{i} n_i {p_i'}(x) \prod_{j \neq i} p_j(x) = \cr
&\sum_{-i} n_i {p_i'}(x) \prod_{-j \neq -i} p_j(x) \prod_{+k} p_k(x) +
\sum_{+i} n_i {p_i'}(x) \prod_{+j \neq +i} p_j(x) \prod_{-k} p_k(x) \cr}$$
Substituting into equation~\ref{renum} yields
\begin{equation}
\label{bigone}
\eqalign{
&{y'} \prod_{-i} p_i(x)^{-n_i+1} = \cr
&{x'} \( \prod_{+i} p_i(x)^{n_i} \)
\sum_{-i} n_i {p_i'}(x) \prod_{-j \neq -i} p_j(x) + \cr
&{x'} \( \prod_{+i} p_i(x)^{n_i-1} \)
\sum_{+i} n_i {p_i'}(x) \prod_{+j \neq +i} p_j(x) \prod_{-k} p_k(x) \cr}
\end{equation}

The right side of this equqtion is now grouped into four polynomial
terms $A {B'} + {A'} B$ where
$$\eqalign{A &=  \prod_{+i} p_i(x)^{n_i} \cr
	   B' &= \sum_{-i} n_i {p_i'}(x) \prod_{-j \neq -i} p_j(x) \cr
	   A' &=  \( \prod_{+i} p_i(x)^{n_i-1} \)
\sum_{+i} n_i {p_i'}(x) \prod_{+j \neq +i} p_j(x) \cr
           B &= \prod_{-k} p_k(x) \cr}$$
$A$ is the original numerator and $A'$ it's derivative.  $B$ and $B'$
can be derived from the squarefree factorization of the denominator of
the integrand.  $A$ and $A'$ can be recovered by a kind of long
division of the right side of equation~\ref{bigone} by $B$ and $B'$
simultaneously.  In addition to subtracting a term times $B'$ subtract
the term's derivative times $B$.

\section{A First Order Differential Equation}

Starting with equation~\ref{product'} multiply through by
$\prod_{i=1}^k p_i(x)$ and replace $\prod_{i=1}^k p_i(x)^{n_i}$ on the
right side by $y$.

\begin{equation}
{y'} \prod_{i=1}^k p_i(x) = {x'} y \sum_{i=1}^k n_i {p_i'}(x)
\prod_{j \neq i} p_j(x)\label{y'y}
\end{equation}

By Lemma~\ref{comfact} this cannot be simplified because the
two sides have no factor in common.  Hence, this form is
canonical.

Therefore, given an equation of form ${y'} q(x) = {x'} {y } r(x)$, if
it can be put into the form of equation~\ref{y'y}, it can be solved as
in equation~\ref{product}.  In order to do this we need to factor
$q(x)$.  This factoring can be seen as the same complexity as the
partial fraction decomposition in Risch's algorithm.

Once we have factored $q(x)$, we need to find a set of $n_i$ so that
$$\sum_{i=1}^k n_i {p_i'}(x) \prod_{j \neq i} p_j(x) = r(x)$$.  Now in
order for this solution to be unique we need to show that the terms
${p_i'}(x) \prod_{j \neq i} p_j(x)$ are lineraly independent and hence
form the basis for a vector space.  Let's assume that they were not
independent.

Suppose there existed a set of integers $m_i$ such that $$\sum_{i=1}^k
m_i {p_i'}(x) \prod_{j \neq i} p_j(x) = 0$$ and there exists some $m_i
\neq 0$.  If only one $m_i \neq 0$ then ${p_i'}(x) \prod_{j \neq i}
p_j(x) = 0$.  Since $p_j(x) \neq 0$ then ${p_i}'(x) = 0$.  But then
$p_i(x)$ would not be a polynomial in x.  So then
\begin{equation}
-m_i {p_i'} \prod_{j \neq i} p_j(x) = \sum_{h \neq i} m_h {p_h'}(x)
\prod_{j \neq h} p_j(x)\label{lindep}
\end{equation}
Now, $p_i(x)$ divides every term on the right side of
equation~\ref{lindep} so $p_i(x)$ must also divide $-m {p_i'}(x)
\prod_{j \neq i} p_j(x)$.  But, because of squarefree, $p_i(x)$ does
not divide ${p_i'}(x)$ and $p_i(x)$ does
not divide ${p_j}(x)$ when $j \neq i$.  Hence, there exists a unique
set of coefficients satisfying equation~\ref{y'y}.

\end{document}
